\section{Background}\label{sec:background}

TODO introduction

\subsection{Analysis of public real-world graph datasets}\label{subsec:graph-analysis}
TODO will include histogram of graph sizes, average outdeegree, maybe clustering coefficient and further interesting metrics, with
regards to the thesis, for (all?) graphs of the SNAP and Labaratory of Web Algorithms dataset collection.

\subsection{Compressed sparse row representation}\label{subsec:csr-background}
Compressed sparse row representation (short CSR) is a well known, low-memory representation for static graphs~\cite{csr,csr-first}.
To ease its explanation, we assume that the graph's vertices are identified by the numbers from 0 to $|V| - 1$.
However, our implementation allows the use of arbitrary vertice identifiers in $\mathcal{N}$ by storing the translation in an additional
array of size \textit{|V|}.

CSR uses two arrays to represent the edge relationship of the graph: one of size \textit{|E|} which is a projection of the edge relationship
onto the \textit{dst} attribute and a second of size \texttt{|V + 1|} which stores indices into the first array.
To find all destinations directly reachable from a source \textit{src $\in$ V}, one accesses the second array at \textit{src} for the
correct index into the first array for a list of destinations.
% TODO maybe example figure?

The CSR format has two beneficial properties in the context of this thesis.
First, it allows locating all destinations for a source vertice by one array lookup;
hence, in constant time.
Second, the representation is only, roughly, half as big than a simple columnar representation.
A uncompressed columnar representation needs $2 \times |E|$ while CSR uses only $|V| + 1 + |E|$, note that for most real-world graph |V|
<< |E| holds (see~\cref{subsec:graph-analysis}).
