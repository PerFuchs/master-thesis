\section{Conclusions} \label{sec:conclusions}

% TODO get back to research questions
% TODO section about timely dataflow comparision, leading question: what's the better way to
% implement a WCOJ, do they need to ship each edge to all nodes, is there a connection to our share proof or
% curse of conncetivity


\subsection{Future work} \label{subsec:future-work}

\subsubsection{Cluster mode} \label{subsubsec:cluster-mode}

\subsubsection{Finer-grained work-stealing} \label{subsubsec:finer-grained-work-stealing}
% TODO does the materialization section include a reference to not moving first level trieIterators
In our experiment section~\ref{subsec:scaling-graphWCOJ}, we noted that work-stealing that
operates only on the first level of variable bindings can still lead to skew for bigger queries.
Therefore, we describe a work-stealing LeapfrogTrieJoin algorithm that allows to steal work
on all levels.
We describe a possible design in two steps.
First, we explain under which circumstances to steal work.
Then, we describe how to steal work.
In the following, we call the process that steals work thief and the other process
victim.

We let each task start with its own bindings for the first variable, e.g. by assigning
a range to each task based on its partition number.
This can be implemented as a range filter in the \textit{LeapfrogJoin} of the first variable.

Once all of these initial bindings are processed, we start stealing work from other
\textit{LeapfrogTriejoins}.
It is beneficial to steal bindings of variables higher in the global order because this
maximizes the amount of work stolen.
A task is encoded as a prefix of variable bindings, e.g. if we steal work in a 5-clique query at
the third level a prefix might be $[4, 1 , 5]$.
If the work-stealing request successfully returns a binding, we set the state of all components
of the current join to the values of the prefix.
Then we run the normal Leapfrog Triejoin algorithm to generate all complete bindings for the stolen
prefix.
When all bindings have been produced, we steal work and repeat the process.

If a work-stealing request cannot find any work, the task finishes.

We face four challenges for the question of how work is stolen.

First, the Leapfrog Triejoin encodes its state in the \textit{TrieIterator} components.
This state should not be changed when we steal work, except for the fact that a stolen
prefix is not considered by the victim.

Second, when we use the \textit{LeapfrogJoin} to steal work it is not guarantued that the underlying
\textit{TrieIterators} are set to the correct level for this \textit{LeapfrogJoin}.

Third, we plan to use shared \textit{LeapfrogJoin} instances to implement work-stealing.
Therefore, this interface needs to become thread-safe.

Fourth, the \textit{LeapfrogJoin} interfaces need to be accessible to the thieves.
As in our current solution, we implement communication via a shared data structure.
This data structure allows access to all \textit{LeapfrogJoin} instances on the same worker.
If a task needs to steal work, it selects one of these \textit{LeapfrogJoin} instances.

We describe the solution to this problems in order.

For the first problem of not changing the state of the \textit{TrieIterators}, we
suggest to add new \textit{seek} and \textit{next} methods to the \textit{LeapfrogJoin} and
\textit{TrieIterator} interface which do not change the state but work exactly as the originals
otherwise.
Additionally, the stateless \textit{next} method of the \textit{LeapfrogJoin} interface stores
the last value it returns.
This value can be used from the stateful method to seek for the upper bound of it, such that
values returned by the stateless version are skipped in the statefull version.
Then thiefs can use the stateless versions and the owner of the interfaces uses the stateful
versions.

The second issue of ensuring that the \textit{LeapfrogJoin} uses the \textit{TrieIterators} with
the correct level is trivially solved by introducing one version per level of the stateless \textit{next}
and \textit{seek} methods of the \textit{TrieIterator} interface and store which to use in the
\textit{LeapfrogJoin}.

Introducing a thread-safe \textit{LeapfrogJoin} interface requires one lock per instance which
needs to be aquired before using any method and released after.
It is not necessary to lock the underlying \textit{TrieIterator} interfaces as we argue in the next
paragraphs.

We start our argument that we only need to consider cases where one thief and one victim
thread interfere with each other.
This is because only one thief can be active at any \textit{LeapfrogJoin} at any time due to the
necessity to hold the lock for this \textit{LeapfrogJoin}.

We observe that to use a specific \textit{LeapfrogJoin} to steal work from, the
victim needs to work on bindings which are later in the variable ordering.
Otherwise, the variables above are not bound and it is not possible to steal a complete prefix.
This is a precondition to be enforced on the thieves side when it chooses a \textit{LeapfrogJoin}
to steal work from.

Then, we require the victim to hold the lock of the respective \textit{LeapfrogJoin}
instance when its enters the \textit{trieJoinUp} method.
This guarantees that the victim \textit{LeapfrogTriejoin} cannot break the assumption above during the process
of work-stealing.

With these assumptions in place, we are ensured that the victim does not call
any \textit{TrieIterator} methods which interfere with the thief.
There are two cases to consider.
First, the \textit{TrieIterator} is independent from the work-stealing process because it is not
part of the intersecion of the \textit{LeapfrogJoin} which is used by the thief.
Second, the \textit{TrieIterator} is used at a deeper level than the one which is used by the
\textit{LeapfrogJoin} to steal work of.
In both cases, the use of the \textit{TrieIterator} does not interfere with the thief.

The changes outlined above allow to share the \textit{LeapfrogJoin} instances of all
tasks of an executor to steal work at any level in the variable ordering.
This should approach should lead to work-stealing jobs which are fine-grained enough
to be nearly skew free for queries of all sizes.
We end the section with a short discussion of lock contention.

The algorithm starts with totally uncontested locks until the first tasks finishes work in its
range.
Then the number of threads which could contend for locks grows linearly with the progress of the
algorithm within finding all possible bindings because before a thread can start work-stealing it
finishes its own range.

When threads start stealing work they can choose their victims such that they minimize lock contention.
Hence, locks are contested mostly between the thief and the victim.
For this case, we note that the thief chooses \textit{LeapfrogJoins} as high in the variable ordering
as possible which are less often used by the victim which spends most of its time with bindings
for lower variables.

Finally, the locking time of the locks is short most of the times because it only needs to find
one further binding.
With materialized \textit{LeapfrogJoins} the locked code section only
reads one value from an array, one or at most two \textit{seek} calls on the first level of an
\textit{TrieIterator}\footnote{These calls need two array reads. Hence, they are actions in constant time.} and stores the value returned.
