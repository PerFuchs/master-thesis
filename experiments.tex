\section{Experiments}

TODO introduction

\subsection{Setup}

\subsubsection{Datasets}
We run our the majority of our experiments on two datasets from different use-cases, social networks and product purchase.
We motivate our choice in the next paragraph.
\Cref{table:datasets} includes a whole list of all graph datasets mentioned throughout the thesis.

\begin{table}[]
    \begin{tabular}{lllll}
        Name    & Variant & Vertices & Edges & Source          \\
        SNB     & sf-1    &          &       & \cite{snb}      \\
        Amazon  & 0302    &          &       & \cite{snapnets} \\
                & 0601    &          &       & \cite{snapnets} \\
        Twitter & sc-d    &          &       & \cite{snapnets} \\
                & sc-u    &          &       & \cite{snapnets}
    \end{tabular}
    \caption{Summary of all datasets mentioned in the thesis.
      Explanation of them and for the variants is given in running text.
    }
    \label{table:datasets}
\end{table}

% TODO add Vertices and edges numbers

The SNB benchmark~\cite{snb} generates data emulating the posts, messages and friendships in a social network.
For our experiments we use only the friendships relationship (\texttt{person\_knows\_person.csv}) which is an undirected relationship.
After generation only edges of the kind \textit{src $\le$ dst} exist, we generate the opposing edges before loading the dataset, such that the edge table we run our experiments
on is truely undirected.
The benchmark comes with an extensively parameterizable graph generation engine
which allows us to experiment with sizes as small as 1GB and up to 1TB for big experiments and different levels of selectivity.
The different sizes are called scale-factor or \texttt{sf}, e.g. \texttt{SNB-sf-1} refers to a Social network benchmark dataset generated with
default parameters and scale-factor 1.
We include the exact parameter used for generation in our repository under \texttt{experiments/snb/params.txt}. % CODEREF TODO include

The Amazon co-purchasing network contains edges between products that have been purchased together and hence are closely related to each other~\cite{snapnets}.
This is a directed relationship from the product purchased first to the product purchased second, both direction of an edge can exist if the order in which
products have been purchased varies.
The Snap dataset collection contains multiple Amazon co-purchase datasets each of them containing a single day of purchases.
We choose the smallest and biggest dataset from the 2nd of March and the 1st of June, we call them \texttt{Amazon-0302} and \texttt{Amazon-0601}.
We pick them for evaluation because former work often concentrated on social networks and web crawl based graphs~\cite{myria-detailed,ammar2018distributed}
while~\cite{salihoglu2018} points out that the biggest graphs are actually graphs graphs like the aforementioned Amazon graph containing purchase information.

To allow comparisions with former work, we run a subset of our experiments on the Twitter social circle network from~\cite{snapnets}.
This dataset includes the follower relationship of one thousand twitter users each of these follows 10 to 4.964 other users and relationship between these are included.
The graph is originally directed but for some experiments we add reversed edges to make the graph undirected - again for comparision with former work.
We call this graph \texttt{Twitter-sc-d} and \texttt{Twitter-sc-u} for the directed respectively undirected variant.
