\section{Related Work} \label{sec:related-work}


% papers:
% Have to
% fractal

% timely dataflow
% longbin

% Maybe
%   CAPs?
%   semih's work on binary plus other joins

% TODO eventually include binary search papers

\subsubsection{Fractal a graph pattern mining system on Spark}

%However, it has been shown that one can use multiple rounds of Shares to trade-off more shuffle operations for less replication while maintaining optimal communication costs~\cite{shares-multi-round}.
%Lately, in 2018, Afrati et al. published a Shares version which can handle skew and implemented it in Hadoop~\cite{sharesSkew,hadoop}.
\subsection{\textsc{WCOJ} on Timely Data Flow}\label{subsec:wcoj-timely-data-flow}

% LOG 1.3.
A second distributed version of worst-case optimal joins was published in 2018 based on a Timely Data Flow system~\cite{ammar2018distributed,naiad}.
In Timely Data Flow, shuffling is a streaming, asynchronous operation which requires no disk access\footnote{The most commonly used cluster computing
engines (Hadoop and Spark) implement shuffling as a synchronizing operation that requires to write and read all tuples from disk.}.
Therefore, the number of shuffle operations is less important than in Hadoop or Spark.
The authors take advantage of this fact by using a uniform, non-replicating partitioning scheme for their relationships.
Then they implement a worst-case optimal join using distributed data flow operators~\cite{naiad}, e.g. min and intersection.
Similar to us, the authors focus on scalability and efficiency in their work but due to the use of a streaming,
in memory shuffles and a fine-grained batched processing scheme their approach is unlikely to be successful in Spark.
Hence, their and our research share the same goals, however, we aim to achieve it in a more restrictive, but widely used and industrial accepted, system.

% Index structure
% same index structure as us, forward and backward index of the whole graph but additionally memory pressure due to prefixes,
% tackled by batches but this is one more thing to configure
% they store the whole index on all workers but could evenly divide it.

% Algorithm explained, concentrate on BigJoin but mention other two
% description of how algorithm works

% three algorithms: BiGJoin static workload, Delta-BigJoin insertion only (communication optimal),
%BigJoin-S (not implemented(
% BigJoin is cumulative worst-case optimal and on real-world datasets gives good workbalance and low
%per worker memory, but inputs that let a single worker do most of the work exists
% BigJoin-S guaruntues worst-case optimal computation, communication, work-balance on all queries, also guarutus
% IN/w memory cost per worker


% Our guaruantues?
% computational given by LFJT
% communication cost, none after broadcast (O(IN)), w \times IN
% memory costs: same as communication costs


% how much better can they do by distributing their indices?
% currently w \times IN, complete index on each worker
% they can reach IN \ w when using the same hash function for each variable --> heavy skew
% different hash functions lead to \sum_v IN \ w - overlap

% N = v, p = (1/w) (assigned), k=0 (no succeess) show for increasing v, gives percentages of IN per worker given w

% What are their communication costs?

% Not for Spark
% in spark communication rounds of v(2r+1), quite high
% compare with binary joins
% batching engineering challenge
% also broadcasted indices, at least prepared indices on all workers
% Spark is none streaming

% General comparision?
% rather not
% but if so, state tracked, number of prefixes in the system at any time,
% our system is monolith, theirs is built of simple operators

% Experiments
% Scaling of BigJoin not given, cannot be compared
% Single threaded on twitter graph (big one) (need to double check which) and LJ, LJ needs 6.5s
% BigJoin 8 workers 16 cores each, takes 3.4 s to find all triangles in LJ
% BigJoin 10 machines 16 cores each 4-clique, house, 5-clique. they do not report dataset, maybe i can find the dataset int
% seed paper
% no experiment regarding communication costs

% Implemenation: https://github.com/frankmcsherry/dataflow-join



% read comparision with Shares

% TODO longbins paper




\subsection{Semih's work on worst-case optimal join for different queries}\label{subsec:wcoj-binary-joins}
% TODO Semih's work
%Below we analyse the scalability of \texttt{HC} in the context of analytical graph queries.
%First, we point out some simple implications of our use-case, then we analyse how a growing number of variables influences the number of tuples per server, the number of servers needed to build the hypercube and the number of replicated tuples in the system.

%We are discussing multi-join processing in the context of graph pattern matching; this implies that variables correspond to the number of nodes in the pattern and relations to the number of edges.
%Hence, we have at least as many relations in each join as we have variables, so the pattern is connected, e.g. as many relations as variables for a path query or $n * (n - 1) / 2$ relations for a n-clique.
%Another implication is that the join ranges only over a single relationship: the edge relationship of the graph.
%When all input relationships are of the same size, the optimal choice for shares in \texttt{HC} are $p_1 = ... = p_{A-1}$~\cite{myria-detailed}.
%In the following, we call this number the size $s$ of the hypercube.
%The third ramification of our use-case is that each relationship has exactly two attributes.
%This means each tuple has two fixed components in its coordinate while the rest is unbounded.
%In the next paragraph, we show how this knowledge helps us to understand how many tuples are sent to each node.


%Each node receives $R * |E| / s^2$ tuples with $R$ the number of relationships in the join.
%Our argument is that each relationship $r_i$ is divided onto $s^2$ nodes (the workers that form the planes of its two attributes) and all $r_i$ are of size $|E|$.
%If a node holds $|E|$ or more tuples after the shuffle, it would have been as good or better to broadcast \textit{E} to all nodes.
%Hence, $s^2$ needs to be bigger than \textit{R} and therefore bigger than \textit{A} (see implications above) to beat the communication costs of a shuffle in our case.
% Extent to unique tuples only



%\subsection{Adaptive Query Exectution}

%A novel development in Spark is the ability to generate code to execute queries on the fly, called WholeStage codegeneration, based on a technique used in the Hyper database~\cite{hyper,jira-whole-stage,1m-rows-laptop}.
%Compiled queries have been shown to be multiple magnitudes faster than interpreted queries traditionally used in most database systems.
%Interpreted queries are most commonly implemented using the Volcano model~\cite{volcano}.
%This model provides a simple and composable interface for algebraic operators; basically every operator would provide an iterator interface.
%This interface would be used by a query by calling next on the root operator, who in turn calls next on each of its children and so on, until the next calls reach the scanning operators at the bottom of the query execution tree. 
%These would provide a single tuple which would then be "pulled" upwards through the query tree and processed by all operators. 
%When it reaches the root operator, the result is delivered to the user. 
%This happens for every tuple; hence, the approach can be described as tuple-at-a-time. % TODO wording
%Although, this interface is simple yet powerful due to its composability, it is also quite computation intensive mainly due to the high number of calls to the next function, which is often a virtual function call.
%This high number of virtual function call is not only CPU intensive but also makes bad use of CPU registers (they are spilled on every function call) and hinders compiler optimizations.
%Compiled queries avoid these costs by generating code specific to each query consisting mainly out of multiple, tight for-loops following each other.
%This speeds up processing by keeping data in the CPU registers as long as possible and avoiding materilization and function calls.
%Furthermore, it allows compiler optimizations, such as loop unrolling or ~\cite{hyper}% TODO one more.
%We are not aware of any published efforts to speed up worst-case optimal joins via code generation.

%We aim to combine the research on worst-case optimal join algorithm and Spark's extensible optimizer Catalyst to speed up graph processing for all Spark users.
%In particular, this work will be based on either of the two distributed versions of worst-case optimal join algorithms mentioned above. 
%We hope to further their work by evaluating which approach (shuffle + local join or timely data flow) works best on a MapReduce based processing engine as well as proving that worst-case optimal join algorithms
%can improve performance on a complex, optimized, existing platform that has not been built with them in mind originally, albeit their high additional cost (e.g. for sorting and need for special data structures).
