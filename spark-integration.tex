\section{Spark integration}\label{sec:spark-integration}

\subsection{User interface} \label{ssec:user-interface}
\begin{listing}[H]
    \centering
    \inputminted{scala}{code/usage-example.scala}
    \caption{Example usage of a WCOJ to find triangles in graph.}
    \label{lst:usage-example}
\end{listing}

As one can see in line 16 % LINE
of~\cref{lst:usage-example}, we support a clean and precise DSL to match patterns in graphs.
This DSL is inspired by GraphFrames~\cite{graphframe}.
The user can define a pattern by its edges, each edge is written as \textit{(a) - [] -> (b)} where \textit{a} is the
source vertice and \textit{b} is the destination, multiple edges are separated by a semicolon.
A connected pattern is expressed by defining multiple edges with the same source or destination.
One should be aware, that a named source or destination is not guaranteed to be a distinct element in the graph,
e.g. \textit{(a) - [] -> (b); (b) - [] -> (c)} could be a linear path of size two or a circle between \textit{a} and
\textit{b}; in the second case \textit{a} and \textit{c} are the same element.
The reader might wonders, why we chose to stay with the GraphFrame syntax for edges of
\textit{- [] ->}, although, we could have went with something simpler, like \textit{->}.
However, sticking to the more verbose syntax allows us to include labels inside of the squared brackets
in future extensions, e.g. for our stretch goal of integration with CAPS.

The second parameter to \textit{findPattern} allows the user to specify the variable ordering used in the WCOJ algorithm.
Furthermore, the user interface takes multiple optional arguments, e.g. to apply to common filters to the output of the result,
specify different relationships to be used as input for each edge of the pattern.
The filters are \textit{distinctFilter}, ensuring that each vertice can occur only as binding for one variable, and
\textit{smallerThanFilter} to allow only output bindings were the values decrease with regards to the specified variable ordering,
e.g. the binding \textit{[1, 2, 3]} but not \textit{[2, 1, 3]} for the triangle query above.
We experienced that these queries are typical for graph queries and that the performance greatly benefits from pushing
them into the join.
Implementing the possibility to push general filters into the join would be a valuable addition but we decided against it because
it a pure engineering task.

\subsection{Integration with Catalyst} \label{ssec:integration-with-catalyst}
% make clear catalyst ist the only thing we change


We integrated our WCOJ implementation into Spark such that it can be used as function on \textit{Datasets}.
Therefore, we build all components necessary to execute a WCOJ in Spark's structured queries, provided by Catalyst (see
\cref{subsubsec:catalyst}).
First of all this is logical plan representing a worst-case optimal join.
Then a strategy to convert this logical operator into multiple physical operators.
One physical operator executes the join.
In between, this operator and the graph edge relationships, we execute another physical operator that materializes
the graph edge relationships into a data structure that can support a \textit{TrieIterator} interface
(see~\cref{subsubsec:leapfrog-triejoin}).
The integration itself is quite straightforward due to Catalysts extendability.
We explain it in the next sections.
First, we highlight some limitations of our integration into Catalyst.

It is not in the scope of this work to integrate WCOJ's into the SQL parser of Spark.
Hence, WCOJ can only be used by Spark's Scala functional interface and not through Spark's SQL queries.

We do not integrate it into the query optimization components of Catalyst, e.g. we do not provide rules or cost-based strategies to
decide when to use a WCOJ or a binary join.
It is up to the user to decide when to use a WCOJ or a binary join.
However, our integration allows the user to intermix these freely.
The reason for this decision is, that at time of writing no published paper existed that systematically studies which queries benefit from
WCOJ's in general, nor, does research exist that studies the combination of WCOJ's and binary joins.
Only a month after we decided on our scope for Spark integration Salihoglu et al. published an arXiv paper~\cite{mhedhbi2019} that
tackled these problems for the first time.
The lack of peer-reviewed papers and the high complexity of the arXiv paper clearly illustrate that deeper integration with
Spark's optimizer is out of scope for this thesis.


\subsection{A sequetial linear Leapfrog Triejoin} \label{subsec:spark-integration-seq}
For this section we assume that the reader is familiar with the background section about Catalyst (see \cref{subsubsec:catalyst}) and
\textsc{LFTJ} (see \cref{subsubsec:leapfrog-triejoin}) where we explain the components of Catalyst planning phase and the requirements
of a Leapfrog Triejoin.
In the current section, we outline how to satisfy the requirments of \textsc{LFTJ} within and help of Catalysts structured plans.

Our baseline implementation of the Leapfrog Triejoin is a sequential implemenation, i.e. it is not distributed.
Therefore, all representations of the edge relationship have only a single partition which the join operates upon.
In Spark this partitioning is called \textit{AllTuples}.
We enforce sequential execution of the complete Spark plan in our experiments by setting the number of executors to 1.

In the first phases of Catalyst query compilation process, the query plan is represented by logical operators.
Integration into this phase only requires us to build a logical operator to represent the \textsc{WCOJ} join.
A \textsc{LFTJ} can have 2 or more children, one for each input relationship.

The logical and physical plan for the triangle query is shown in~\cref{fig:wcoj-catalyst-plan}.

\begin{figure}
%    \subfloat[Logical plan]{\includesvg[width=0.5\textwidth]{wcoj-catalyst-logical-plan}}
%    \subfloat[Physical plan]{\includesvg[width=0.5\textwidth]{wcoj-catalyst-physical-plan}}
    \caption{Catalyst plans for the triangle query using a Leapfrog Triejoin.}
    \label{fig:wcoj-catalyst-plan}
\end{figure}

The strategy to translate the logical plan into a physical plan has to tasks.
First, simply translating the n-ary logical plan into n-ary physical plan that executes the \textsc{LFTJ} with the children as input.
Second, introducing a physical operator per child which materializes the RDD into a sorted, columnar array representation to support a
\textit{TrieIterator} interface.

The first physical operator is straightforward to implement.
It simply executes a \textit{LFTJ} over the \textit{TrieIterators} provided by the children.
Given that each child has only one partition and there are no parallel operations, the algorithm can be implemented exactly
as described in~\cref{subsubsec:leapfrog-triejoin}.

The second physical operator translates the linear interator interface of any RDD into a \textit{TrieIterator} interface.
In particular, it needs to offer a \textit{seek} operation in $\mathcal{O} (\log N) $.
To support this interface the operator requires its children to be sorted;
this requirement can be fulfilled by Catalyst standard optimization rules.
Then it takes the sorted linear iterator and materializes it into a column-wise array structure.
Given this data structure, the \textit{TrieIterator} can be implemented using binary search (as described
in~\cref{subsubsec:leapfrog-triejoin}).

\subsection{GraphWCOJ} \label{subsec:spark-integration-graphWCOJ}
% TODO maybe i miss a good connection between variable ordering and sorting and the fact that I need only two tables?
The integration of GraphWCOJ is quite similar to the one of \textsc{LFTJ}.
\Cref{fig:graphWCOJ-catalyst-plan} shows the logical and physical plan constructred by our integration.
There are three main difference: GraphWCOJ requires only two children for the input relationships, the children materialize the input
relationships in a CSR data structure (see~\cref{subsec:csr-background}) and we support parallel execution of the join (which requires
a third child) and broadcasting the CSR data structure.
We address these differences in order.

Our GraphWCOJ operators need only two materialized version of the input relationship.
This is because in graph pattern matching the joins are self-joins on the edge relationship.
The \textsc{LFTJ} requires that its input relationships are sorted by an lexicographic sorting over the variable ordering.
To support all possible variable orderings, we need the edge relationship sorted by $src, dst$ and $dst, src$.
Hence, we need to separate, materialized versions of the edge relationship.

GraphWCOJ uses a CSR representation of the edge relationship (see~\cref{sec:graphwcoj}).
% TODO CSR accessible and sorted? definition not given.
Hence, we need to build two CSR representations.
One with its \textit{Indices} array build from the $src$ attribute and the \textit{AdjancencyLists} array build from the $dst$ attribute.
The other, with \textit{Indices} from $dst$ and \textit{AdjacencyLists} from $src$.
Next, we describe how to build these CSR's from two linear, sorted, row wise iterators as provided by Spark for the two child relationships.

% TODO algorithm to build them
First, we note that it is necessary to build both compressed sparse row data structures in tandem.
This is because some vertices in the graph might have no outgoing or incoming edges.
That means some vertice ID's do not occur in the $src$ or $dst$ attributes of any tuple.
Therefore, the \textit{Indices} arrays of the two CSR structures would differ if they are build from either $src$ and $dst$, e.g.
they could have different length.
However, if this is the case, it is not possible to use both CSR's together in a single join.

To allow building the CSR's in-tandem, we introduce \textit{AlignedZippedIterator}.
The \textit{next} method of this iterator is shown in~\cref{alg:alignedZippedIterator}.
It zips two iterators of two-tuple elements and aligns them on the first component, e.g. edges with $src$ as first component and $dst$ as
second component.
The zipped iterator emits triples where the first component is the aligned first element of both underlying iterators and the other
two elements are the second components of both iterators.
If the two iterators have different numbers of elements with the same first component, we advance only one iterator and fill the missing
component in the emitted triple with a placeholder until the first component of both iterators aligns again.

\begin{algorithm}
  \uIf{iter1.hasNext() and iter2.hasNext()} {
    t1 $\leftarrow$ iter1.peek() \;
    t2 $\leftarrow$ iter2.peek() \;
    \uIf {t1[0] == t2[0]} {
      iter1.next()\;
      iter2.next()\;
      \KwRet{(iter1.key()[0], iter1.key()[1], iter2.key()[1])}\;
    } \uElseIf{t1[0] < t2[0]} {
      iter1.next()\;
      \KwRet{(iter1.key()[0], iter1.key()[1], -1)} \;
    } \Else {
      iter2.next()\;
      \KwRet{(iter2.key()[0], -1, iter2.key()[1])} \;
    }
  } \uElseIf {iter1.hasNext()} {
    iter1.next() \;
    \KwRet{(iter1.key()[0], iter1.key()[1], -1)} \;
  } \Else {
    iter2.next() \;
    \KwRet{(iter2.key()[0], -1, iter2.key()[1])} \;
  }
  \caption{\textit{next} method of an \textit{AlignedZippedIterator}.}
  \label{alg:alignedZippedIterator}
\end{algorithm}

Given an \textit{AlignedZippedIterator} over both input relationships, it is straightforward to build both CSR data structures.
We consume the whole \textit{AlignedZippedIterator}, for each element we append the 2nd and 3rd component to \textit{AdjacencyLists} of
the CSR's;
while skipping placeholders.
Whenever the first element of the three tuple changes, we append the current size of the \textit{AdjaencyLists} buffers to the
\textit{Idices} arrays.

The final difference between the Spark integration for \textsc{LFTJ} and GraphWCOJ is that we build GraphWCOJ such that it can
be run in parallel.

As argued in former chapters, we broadcast the edge relationship to all workers.
The broadcast is supported by Spark's broadcast variables (see~\cref{subsubsec:broadcast-variables}) and Catalysts support to broadcast
the execution of a physical operator.

Parallelism is introduced via the third child of our GraphWCOJ operators.
It is an empty RDD with as many partitions as the desired level of parallelism.
We schedule tasks by using the \textit{mapPartitions} function of this empty RDD.
Foreach partition, we run the \textsc{WCOJ} join backing its \textit{TrieIterator} with the broadcasted CSR's and partition the data
logical by one of the schemes described in~\cref{sec:worst-case-optimal-join-parallelization}.


% Broadcast cache
One of the main advantages of broadcasting the edge relationship to all workers is that we can reuse the same broadcast for all queries
over the same graph.
To support this in Catalyst, we introduce one additional physical operator which we call \textit{ReusedCSRBroadcast} and a CSR broadcast
variable cache maintained on the Spark master node.

The CSR broadcast variable cache is a simple dictionary with RDD ID's to broadcast variables of CSR structures.
When, our system builds a broadcasted CSR structure, it registers the broadcast in the cache.
Everytime, we translate a logical WCOJ plan into a physical one, we check if the CSR for edge relationship has been broadcasted already
if so, our strategy reuses this broadcast.


% TODO disk cache?
% CSR disk cache  % CSRTrieIterableBroadcast implements that

% connection to distribution, logical partitionings integrated into the GraphWCOJ algorithm


% TODO where whole system overview as one big diagram, and then used to explain the structure? introduction, extra chapter after background.

% TODO move to acknowledgements
The Spark integration of this work has been inspired by the IndexedDataframes~\cite{indexed-dataframes} work of Alex Uta and the supervisors
of this thesis.


